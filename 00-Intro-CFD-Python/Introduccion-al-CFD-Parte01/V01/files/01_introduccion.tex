\begin{frame}{Importancia de la Modelación numérica}
Al día de hoy no podemos negar que entender y saber de manera anticipada los eventos que van a ocurrir ayuda mucho al campo técnico en las ciencias e ingenierías y con el avance computacional que se ha desarrollado es importante que el profesional esté preparado para abordar a estos problemas, mediante herramientas de cálculo computacional el profesional puede reproducir o prevenir los acontecimientos en estudio, para nuestro curso este pretende dar las pautas iniciales para que el participante pueda orientar su camino en su formación, mostrando algunas de las fórmulas utilizadas en este estudio y también compartir código.
\end{frame}

\begin{frame}{Campos de aplicación}
\begin{itemize}
\item Análisis de resistencia de fallas mecánicas.
\item Polución de contaminantes en la atmósfera.
\item Comportamiento de las corrientes de viento en los alabes de los aerogeneradores.
\item Estudiar la capa límite en las ciudades a nivel regional.
\item Por ejemplo el flujo de sangre en un arteria.
\end{itemize}
\end{frame}


\begin{frame}{Python nuestro lenguaje de programación}
Python es un lenguaje interpretado y ello lo hace genial, tiene potencial para las las ciencias e ingenierías:
	\begin{itemize}
	\item Es un lenguaje multipropósito.
	\item Sintáxis clara y fácil de aprender.
	\item Programación orientada a objetos.
	\item Lo puedes usar como pegamento con otros lenguajes de 
	programación.
	\item Es open-source
	\item Multiplataforma para (Linux, Unix, WIndows y Mac Os).
	\end{itemize}
\end{frame}

\begin{frame}{Historia}
Python fue desarrollado en 1991 por el Holandez Guido van Roussun a finales de los años 80's, el nombre esta inspirado en el grupo humorista de la televisión Monty Python's Flying Circus, desde ese año al día de hoy ha evolucionado bastante, sobre todo ha desarrollado una comunidad realmente gigantezca, esa creo es su principal fortaleza, hay librerías para casi todo y sino lo puedes crear y tu mismo subir a la Internet, es muy demandado por las empresas.
\end{frame}

